%\documentclass[twocolumn]{article}
\documentclass[./chem_exercises.tex]{subfiles}
\begin{document}
\begin{multicols}{2}
%\begin{titlepage}
%\maketitle
%\end{titlepage}
\textit{\textbf{Chapter 15 Acids and Bases - summary} }\\

\textit{\textbf{ 15.1 Br\o nsted Acids and Bases} }\\

Br\o nsted acids donate protons, and Br\o nsted bases accept protons.\\

\textit{\textbf{ 15.2 The Acid-Base properties of water} }\\
\begin{flalign*}
K_W &=[\ch{H}^+][\ch{OH}^-]=1.0\cdot 10^{-14}
\end{flalign*}

\textit{\textbf{ 15.3 pH - A measure of Acidity} }\\
\begin{flalign*}
\ch{pH} &=-log[\ch{H3O}^+]=-log[\ch{H}^+]\\
[\ch{H}^+]&=10^{-pH}\\
\end{flalign*}
Similarly
\begin{flalign*}
\ch{pOH} &=-log[\ch{OH}^-]\\
[\ch{OH}^-]&=10^{-\ch{pOH}}
\end{flalign*}
If we take the negative logarithm of the ion-product constant
\begin{flalign*}
[\ch{H}^+][\ch{OH}^-]&=1.0\cdot 10^{-14}\\
-log[\ch{H}^+][\ch{OH}^-]&=-log 1.0\cdot 10^{-14}\\
-log[\ch{H}^+]-log[\ch{OH}^-]&=14\\
\ch{pH}+\ch{pOH} &= 14
\end{flalign*}
At $25^o$C, an acidic solution has pH $<$ 7, a basic solution has pH $>$ 7 and a neutral
solution has pH = 7.\\

\textit{\textbf{ 15.4 Strength of Acids and Bases } }\\
Strong acids are stron electrolytes that for practiccal purposes
are assumed to ionize completely in water.\\
\begin{flalign*}
\ch{HCL}(aq) +\ch{H2O}(l) &\rightarrow \ch{H3O}^+(aq) + \ch{Cl}^-(aq)\\
\ch{HNO3}(aq) +\ch{H2O}(l) &\rightarrow \ch{H3O}^+(aq) + \ch{NO3}^-(aq)\\
\ch{HClO4}(aq) +\ch{H2O}(l) &\rightarrow \ch{H3O}^+(aq) + \ch{ClO4}^-(aq)\\
\ch{H2SO4}(aq) +\ch{H2O}(l) &\rightarrow \ch{H3O}^+(aq) + \ch{HSO4}^-(aq)\\
\end{flalign*} 

Strong bases are strong electrolytes  and ionize completely in water.
Hydroxides of alkalimetals and certain alkaline earth metals are strong bases.\\
\begin{flalign*}
\ch{NaOH}(s)&\xrightarrow{\ch{H2O}}\ch{Na}^+(aq)+\ch{OH}^-(aq)\\
\ch{KOH}(s)&\xrightarrow{\ch{H2O}}\ch{K}^+(aq)+\ch{OH}^-(aq)\\
\ch{Ba(OH)2}(s)&\xrightarrow{\ch{H2O}}\ch{Ba}^{2+}(aq)+2\ch{OH}^-(aq)\\
\end{flalign*} 
Weak bases, like weak acids, are weak electrolytes. Ammonia is a weak base. It ionizes
to a limited extent in water
\begin{flalign*}
\ch{NH3}(aq)+\ch{H2O}(l) \ch{<=>}\ch{NH4}^+(aq) + \ch{OH}^-(aq)\\
\end{flalign*} 
\textit{\textbf{ Example 15.7 } }\\
Predict the direction of the following reaction in aqueous solution
\begin{flalign*}
\ch{HNO2}(aq)+\ch{CN}^-(aq) \ch{<=>}\ch{HCN}(aq) + \ch{NO2}^-(aq)\\
\end{flalign*} 
The problem is to determine whether at equilibrium, the reaction will be shifted
to the right, favouring \ch{HCN} and \ch{NO2^-} or to the left favouring
\ch{HNO2} and \ch{CN^-}. Which of the two is the stronger acid and hence the stronger
proton donor \ch{HNO2} or \ch{HCN}? Which of the two is the stronger base and hence a stronger
proton acceptor \ch{CN^-} or \ch{NO2^-}?Remember that the stronger the acid the weaker its conjugate
base.\\

In Table 15.2 we see that \ch{HNO2} is a stronger acid than \ch{HCN}. Thus \ch{NO2} is weaker than
\ch{CN^-} which means that \ch{CN^-} is a stronger proton acceptor and \ch{HNO2} is a stronger proton donor, 
thus the net reaction will proceed from left to right.\\

\end{multicols}
\vfill\null
\clearpage
\textit{\textbf{ 15.5 Weak Acids and Acid Ionization constants } }\\
Consider
\begin{flalign*}
\ch{HA}(aq)+\ch{H2O}(l) \ch{<=>}\ch{H3O^+}(aq) + \ch{A}^-(aq)\\
\end{flalign*} 
or simply
\begin{flalign*}
\ch{HA}(aq) \ch{<=>}\ch{H3O^+}(aq) + \ch{A}^-(aq)\\
\end{flalign*} 
The equilibrium expression
\begin{flalign*}
K_a &=\frac{[\ch{H3O^+}][\ch{A}^-]}{[\ch{HA}]}
\end{flalign*} 
where $K_a$, the \textit{acid ionization constant}, is the equilibrium constant for ionization
of an acid. The larger $K_a$ the stronger the acid.

\begin{center}
\begin{tabular}{c c c c c c} 
  & \ch{HF}(aq) & \ch{<=>} &\ch{H^+}(aq)& + &\ch{F^-}(aq) \\ 
Initial(M): &0.50   &  &0.00&  &0.00\\ 
Change(M):  &-x     &  &x   &  &x\\ 
\hline
Equilibrium(M): &0.50-x  &  &x&  &x\\
\end{tabular}
\end{center}
%\hspace{1em}\\
\begin{multicols}{2}
From the table we get $K_a=7.1\cdot 10^{-4}$
So we can express the equilibrium as
\begin{flalign*}
K_a                 &=\frac{[\ch{H3O^+}][\ch{F}^-]}{[\ch{HF}]}\\
7.1\cdot 10^{-4}    &=\frac{(x)(x)}{0.50-x}\\
\end{flalign*} 
rearranging the expression we get
\begin{flalign*}
x^2-7.1\cdot 10^{-4} (0.50-x)&=0\\
x^2+7.1\cdot 10^{-4}x-3.55\cdot 10^{-4}&=0\\
\end{flalign*} 
Because HF is a weak acid and weak acids ionize only to a slight extent, we reason that $x$
must be small compared to $0.50$. Therefore we can make the approximation
\begin{flalign*}
0.50-x\approx 0.5
\end{flalign*} 
so we get the equation
\begin{flalign*}
7.1\cdot 10^{-4} &=\frac{(x)(x)}{0.50}\iff\\
x^2&=0.50\cdot 7.1\cdot 10^{-4}=3.55\cdot 10^{-4}\\
x&=\sqrt{3.55\cdot 10^{-4}}=0.019 \text{ M}
\end{flalign*}
Thus we have solved for $x$ without solving the quadratic equation
We have
\begin{flalign*}
[\ch{HF}]&=(0.50-0.019)M = 0.48M\\
[\ch{H^+}]&=0.019M\\
[\ch{F^-}]&=0.019M\\
\end{flalign*} 
and the pH of the solution is
\begin{flalign*}
pH&=-log[\ch{H^+}]\\
&=-log 0.019 = 1.72
[\ch{H^+}]&=0.019M\\
\end{flalign*}

\textit{\textbf{ 15.9 Molecular Structure and the Strength of Acids } }\\ 

Oxoacids having different central atoms that are from the same group of the periodic table 
and that have the same oxidation number, within this group acid strength increases with incresing
electronegativity.\\

Oxoacids having the same central atom but different numbers of attached groups. Within this group, acid strengths
increases as the oxidation number of the central atom increases.\\


\end{multicols}


%\vfill\null
%\clearpage
%\columnbreak
%\newpage



%\underbrace{}

% \hspace{1em}

%\begin{enumerate}[label=(\alph*)]
%\end{enumerate}

%$$
%  A = 
%  \begin{bmatrix}
%    1 & 0  & 2i\\
%    2i & 0 &  -4\\
%    -i &  0 & -2i\\
%  \end{bmatrix}
%$$

%\begin{flalign*}
%  A = 
%  \begin{bmatrix}
%    1 & 0  & 2i\\
%    2i & 0 &  -4\\
%    -i &  0 & -2i\\
%  \end{bmatrix}
%\end{flalign*}


%\begin{flalign*}
%\psi(x) = \begin{cases} Ae^{ikx}+Be^{-ikx} &\ \  x<-a \\
%                        Ce^{\kappa x}+De^{-\kappa x} &\ \ -a < x < a\\
%						Fe^{ikx} & \ \ x>a
%       \end{cases}
%\end{flalign*}

%\begin{figure}[H]
%  \includegraphics[width=\linewidth]{odd_finite.eps}
%  \caption{$z_0=0.1\pi,0.5\pi, 3\pi,7\pi$}
%  \label{fig4}
%\end{figure}
\end{document}

\vfill\null
\clearpage
\columnbreak
\newpage










                                     
                                     



