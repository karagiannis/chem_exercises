%\documentclass[twocolumn]{article}
\documentclass[./chem_exercises.tex]{subfiles}
\begin{document}

%\begin{titlepage}
%\maketitle
%\end{titlepage}


%\layout

%\chapter{Syntes av Koppar(II)sulfat}
\section{Appendix}
\subsection{Reaktionsformel Di-Koppar(I)Oxid och Salpetersyra}
30 millimol \ch{HNO3} och 30 millimol vatten bildar 30 millimol Nitratjoner och 30 millimol
oxoniumjoner
\begin{flalign*}
\ch{HNO3}+\ch{H2O}\rightarrow \ch{NO3^{-1}} + \ch{H3O^+}&(1)\\
\end{flalign*}
10 millimol Svavelsyra bildar 10 millimol oxoniumjoner och 10 millimol vätesulfatjon
\begin{flalign*}
\ch{H2SO4}+\ch{H2O}\rightarrow \ch{HSO4^{-1}} + \ch{H3O^+}&(1)\\
\end{flalign*}

Försöker ta det i två steg istället
\begin{flalign*}
x_1\ch{Cu(I)2O}+x_2\ch{HNO3}\rightarrow x_3\ch{Cu(II)(NO3)2}+x_4\ch{NO}+x_5\ch{H2O}\\
\end{flalign*}
Ger följande ekvationer
\begin{flalign*}
x_1\cdot 2\ch{Cu(I)}&=x_3\cdot \ch{Cu(II)}\\
x_1 \ch{O} +x_2\cdot 3\ch{O} &= x_3\cdot 6\ch{O} + x_4\cdot \ch{O} +x_5\cdot \ch{O}\\
x_2\cdot \ch{H} &=x_5\cdot 2\ch{H}\\
x_2\cdot \ch{N}&=x_3\cdot 2\ch{N}\\
\end{flalign*}
Strukturerar om och förkortar bort grundämnena
\begin{flalign*}
x_1\cdot 2 -x_3&=0\\
x_1 +x_2\cdot 3 -x_3\cdot6- x_4-x_5&=  0 \\
x_2 -x_5\cdot 2&=0\\
x_2-x_3\cdot 2&=0\\
\end{flalign*}
4 ekvationer och 5 obekanta så därför sätter vi $x_1=1$
\begin{flalign*}
x_1&=1\\
2\cdot x_1  -x_3&=0\\
x_1 +3\cdot x_2 -6\cdot x_3- x_4-x_5&=  0 \\
x_2 -2\cdot x_5&=0\\
x_2-2\cdot x_3 &=0\\
\end{flalign*}
Vilket ger matrisen $A$
\begin{flalign*}   
 A &= \begin{pmatrix}
    1 & 0 &  0 & 0  & 0\\
    2 & 0 & -1 & 0  & 0\\
    1 & 3 & -6 & -1 & -1\\
    0 & 1 &  0 & 0  & -2\\
	0 & 1 & -2 & 0  & 0\\
  \end{pmatrix}\text{ ; } b=\begin{pmatrix}1\\0\\0\\0\\0\end{pmatrix}
\end{flalign*}
Matlab/Octave
\begin{verbatim}
A=[1, 0,   0,  0,  0;
   2, 0 , -1,  0,  0;
   1, 3 , -6, -1, -1;
   0, 1 ,  0,  0, -2;
   0, 1 , -2,  0,  0]
b=[1;0;0;0;0]
x=inv(A)*b
\end{verbatim}
Resultat
\begin{verbatim}
x =

   1
   4
   2
  -1
   2
\end{verbatim}
Så denna reaktionsväg fungerar inte.
Vi prövar 
\begin{flalign*}
x_1\ch{Cu(I)2O}+x_2\ch{HNO3}\rightarrow x_3\ch{Cu(II)(NO3)2}+x_4\ch{NO}+x_5\ch{NO2}+x_6\ch{H2O}\\
\end{flalign*}
som ger följande ekvationer
\begin{flalign*}
2\ch{Cu(I)}x_1 &=1\ch{Cu(II)}x_3 \\
1\ch{O}x_1  + 3\ch{O}x_2 &= 6\ch{O}x_3  + x_4\cdot \ch{O} +2\ch{O}x_5 + 1\ch{O}x_6\\
1\ch{H}x_2  &=2\ch{H}x_6 \\
1\ch{N}x_2 &=2\ch{N}x_3+1\ch{N}+1\ch{N}x5 \\
\end{flalign*}
Strukturerar om och förkortar bort grundämnena
\begin{flalign*}
2x_1 -1x_3&= 0\\
1x_1  + 3x_2 -6x_3-x_4-2x_5-1x_6&=0\\
1x_2 -2x_6 &=0 \\
1x_2 -2x_3-1x_4-1x_5&=0 \\
\end{flalign*}
Vilket ger matrisen där $x_1=1$
Vilket ger matrisen $A$
\begin{flalign*}   
 A &= \begin{pmatrix}
    1 & 0 &  0 & 0  & 0 & 0\\
    2 & 0 & -1 & 0  & 0 & 0\\
    1 & 3 & -6 & -1 & -2& -1\\
    0 & 1 &  0 & 0  & -0 &-2\\
	0 & 1 & -2 & -1  & -1 & 0\\
  \end{pmatrix}\text{ ; } b=\begin{pmatrix}1\\0\\0\\0\\0\\0\end{pmatrix}
\end{flalign*}
För att invertera denna så måste den vara kvadratisk så vi gissar
att $x_6 = 4,5,6,7$ osv. testar först $x_6=4$
\begin{flalign*}   
 A &= \begin{pmatrix}
    0 & 0 &  0 & 0  & 0 & 1\\
    1 & 0 &  0 & 0  & 0 & 0\\
    2 & 0 & -1 & 0  & 0 & 0\\
    1 & 3 & -6 & -1 & -2& -1\\
    0 & 1 &  0 & 0  & -0 &-2\\
	0 & 1 & -2 & -1  & -1 & 0\\
  \end{pmatrix}\text{ ; } b=\begin{pmatrix}4\\1\\0\\0\\0\\0\\0\end{pmatrix}
\end{flalign*}
Matlab/Octave
\begin{verbatim}
A=[0,   0,   0,  0,   0,   1;
   1,   0,   0,  0,   0,   0;
   2,   0,  -1,  0,   0,   0;
   1,   3,  -6, -1,  -2,   -1;
   0,   1,   0,  0,   -0  -2;
   0,   1,  -2,   -1,  -1   0]
b=[4;1;0;0;0;0]
x=inv(A)*b
\end{verbatim}
Resultat blir negativt för $x_4$ då $x_6\geq 4$ och lika med noll då $x_6=3$ därefter blir
$x_5<0$ för $x_6=2$ och $x_6=1$.
Om vi slutar att använda $x_6$ som parameter och antingen sätter $x_4=0$ eller $x_5=0$. Vad händer då?
Uppenbarligen så kan vi inte samtidigt ha $x_4,x_5 >0$ för rimliga värden på $x_6$.
Börjar med att sätt $x_4=0$
\begin{flalign*}   
 A &= \begin{pmatrix}
    0 & 0 &  0 & 1  & 0 & 0\\
    1 & 0 &  0 & 0  & 0 & 0\\
    2 & 0 & -1 & 0  & 0 & 0\\
    1 & 3 & -6 & -1 & -2& -1\\
    0 & 1 &  0 & 0  & -0 &-2\\
	0 & 1 & -2 & -1  & -1 & 0\\
  \end{pmatrix}\text{ ; } b=\begin{pmatrix}0\\1\\0\\0\\0\\0\\0\end{pmatrix}
\end{flalign*}
Resultatet blir då $(x_1,x_2,x_3,x_4,x_5,x_6)=(1,6,2,0,2,3)$
och om vi sätter $x_5=0$ istället
\begin{flalign*}   
 A &= \begin{pmatrix}
    0 & 0 &  0 & 0  & 1 & 0\\
    1 & 0 &  0 & 0  & 0 & 0\\
    2 & 0 & -1 & 0  & 0 & 0\\
    1 & 3 & -6 & -1 & -2& -1\\
    0 & 1 &  0 & 0  & -0 &-2\\
	0 & 1 & -2 & -1  & -1 & 0\\
  \end{pmatrix}\text{ ; } b=\begin{pmatrix}0\\1\\0\\0\\0\\0\\0\end{pmatrix}
\end{flalign*}
så fas om man multiplcerar lösningsvektorn med faktorn 3
$(x_1,x_2,x_3,x_4,x_5,x_6)=(3,14,6,2,0,7)$
Sammanfattar detta i en tabell
\begin{center}
\begin{tabular}{ |c|c|c| } 
 \hline
  \text{ }            &  $x_4=0$&  $x_5=0$ 	 \\ 
\hline
$x_1\ch{Cu(I)2O}$     &1		&3	\\
$x_2\ch{HNO3}$        &6		&14 \\	
$x_3\ch{Cu(II)(NO3)2}$&2		&6	\\
$x_4\ch{NO}$          &0		&2	\\
$x_5\ch{NO2}$         &2		&0	\\
$x_6\ch{H2O}$         &3		&7	\\
 \hline
\end{tabular}
\end{center}
Sannolikt så frigörs nog vätgas \ch{2} så den verkliga reaktionen ser nog ut så här
\begin{flalign*}
x_1\ch{Cu(I)2O}+x_2\ch{HNO3}\rightarrow x_3\ch{Cu(II)(NO3)2}+x_4\ch{NO}+x_5\ch{NO2}+x_6\ch{H2O}+x_7\ch{H2}\\
\end{flalign*}
vilket ger ekvationerna
\begin{flalign*}
2\ch{Cu(I)}x_1 &=1\ch{Cu(II)}x_3 \\
1\ch{O}x_1  + 3\ch{O}x_2 &= 6\ch{O}x_3  + x_4\cdot \ch{O} +2\ch{O}x_5 + 1\ch{O}x_6\\
1\ch{H}x_2  &=2\ch{H}x_6 +2\ch{H}x_7\\
1\ch{N}x_2 &=2\ch{N}x_3+1\ch{N}x_4+1\ch{N}x5 \\
\end{flalign*}
Strukturerar om och förkortar bort grundämnena
och lägger till parameter-ekvationer för att matrisen ska
bli kvadratisk. Fixerar alltså $x_5$ och $x_7$ till olika heltal och ser
om ekvationsystemet har heltalslösningar där $x_i \geq 0$
\begin{flalign*}
2x_1 -1x_3&= 0\\
1x_1  + 3x_2 -6x_3-x_4-2x_5-1x_6&=0\\
1x_2 -2x_6-2x_7 &=0 \\
1x_2 -2x_3-1x_4-1x_5&=0 \\
                 x_1&=1\\
                 x_5 &=p_1\\
                 x_7&=p_2\\
\end{flalign*}
\begin{flalign*}   
 A &= \begin{pmatrix}
    2 & 0 & -1 & 0  & 0 & 0  & 0\\
    1 & 3 & -6 & -1 & -2& -1 & 0\\
    0 & 1 &  0 & 0  &  0 &-2 &-2\\
	0 & 1 & -2 & -1  & -1 & 0 & 0\\
	1 & 0 &  0 & 0   & 0  & 0 & 0\\
	0 & 0 &  0 & 0   & 1  & 0 & 0\\
	0 & 0 &  0 & 0   & 0  & 0 & 1\\
  \end{pmatrix}\text{ ; } b=\begin{pmatrix}0\\0\\0\\0\\1\\ p_1\\p_2\\\end{pmatrix}
\end{flalign*}
Finner inga positiva heltalslösningar där existens av \ch{H2} tillåts och
\ch{NO_x} finnes då $0\leq p_1\leq 20$ och $p_2$ varieras för varje $p_1$ från
$0 \leq p_2 \leq p_1+10$. Den enda som finns är 
$(x_1,x_2,x_3,x_4,x_5,x_6,x_7)=(1,6,2,0,2,3,0)$
och $(x_1,x_2,x_3,x_4,x_5,x_6,x_7)=(3,14,6,2,0,7,0)$ där lösningsvektorn
har multiplicerats med $3$ såsom tidigare.
\begin{verbatim}
clear all
for p_2=0:1:20;
  p1_min=0;
  p1_max=p_2+10;
  for p_1=p1_min:p1_max;
    b=[0;0;0;0;1;p_1;p_2];
    p_1;
    p_2;
    y=inv(A)*b;
    if((y(1)>=0) && (y(2)>=0)&&(y(3)>=0) &&...
	   (y(4)>=0)&&(y(5)>=0)&&(y(6)>=0)&(y(7)>=0))
     y
     break
     endif
  endfor
endfor
\end{verbatim}
%\vfill\null
%\clearpage
%\columnbreak
%\newpage



%\underbrace{}

% \hspace{1em}

%\begin{enumerate}[label=(\alph*)]
%\end{enumerate}

%$$
%  A = 
%  \begin{bmatrix}
%    1 & 0  & 2i\\
%    2i & 0 &  -4\\
%    -i &  0 & -2i\\
%  \end{bmatrix}
%$$

%\begin{flalign*}
%  A = 
%  \begin{bmatrix}
%    1 & 0  & 2i\\
%    2i & 0 &  -4\\
%    -i &  0 & -2i\\
%  \end{bmatrix}
%\end{flalign*}


%\begin{flalign*}
%\psi(x) = \begin{cases} Ae^{ikx}+Be^{-ikx} &\ \  x<-a \\
%                        Ce^{\kappa x}+De^{-\kappa x} &\ \ -a < x < a\\
%						Fe^{ikx} & \ \ x>a
%       \end{cases}
%\end{flalign*}

%\begin{figure}[H]
%  \includegraphics[width=\linewidth]{odd_finite.eps}
%  \caption{$z_0=0.1\pi,0.5\pi, 3\pi,7\pi$}
%  \label{fig4}
%\end{figure}
\end{document}








\vfill\null
\clearpage
\columnbreak
\newpage










                                     
                                     



