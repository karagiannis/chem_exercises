%\documentclass[twocolumn]{article}
\documentclass[./chem_exercises.tex]{subfiles}
\begin{document}
\begin{multicols}{2}
%\begin{titlepage}
%\maketitle
%\end{titlepage}
\textit{\textbf{Chapter 14 Chemical equilibrium - summary} }\\

\textit{\textbf{ 14.1 The concept of equilibrium and the equilibrium constant - summary} }\\
If we have the following equlibrium
\begin{flalign*}
a\ch{A}+b\ch{B}\ch{<=>}c\ch{C}+d\ch{D}
\end{flalign*}
where $a$, $b$, $c$ and $d$ are the stoichiometric coefficients for the reacting
species A, B, C and D. For the reaction at a particular temperature
\begin{flalign*}
K&=\frac{[C]^c[D]^d}{[A]^a[B]^b}
\end{flalign*}
where $K$ is called the equilibrium constant and is called \textit{law of mass action} which says
that \textit{for a reversible reaction at equilibrium and a constant temperature, a certain ratio
of reactant and product concentrations has a constant value K}.\\

If $K\gg 1$ the equilibrium will lie to the right and favour the products if $K\ll 1$ it will favour the
reactants

\textit{\textbf{ 14.2 Writing equilibrium constant expressions - summary} }\\

\textit{\textbf{ Homogenous Equilibria} }\\
The term homogenous equilibrium applies to reactions in which all reacting species are in the same phase.\\
If we have 
\begin{flalign*}
\ch{N2O4}(g)\ch{<=>}2\ch{NO2}(g)
\end{flalign*}
we can write the equilibrium constant with respect to concentrations as
\begin{flalign*}
K_c&=\frac{[\ch{NO2}]^2}{[\ch{N2O4}]}\\
\end{flalign*}
or in terms of pressure
\begin{flalign*}
K_p&=\frac{P_{\ch{NO2}}^2}{P_{\ch{N2O4}}}\\
\end{flalign*}
Generally if we have the following equilibrium in the gas phase
\begin{flalign*}
a\ch{A}(g)\ch{<=>}b\ch{B}(g)
\end{flalign*}
The equilibrium constants are given by
\begin{flalign*}
K_c&=\frac{[B]^b}{[A]^a}\\
\end{flalign*}
and
\begin{flalign*}
K_p&=\frac{P_B^b}{P_A^a}\\
\end{flalign*}
where
\begin{flalign*}
p_AV&=n_ART\\
p_A&=\frac{n_ART}{V}\\
p_BV&=n_BRT\\
p_B&=\frac{n_BRT}{V}\\
\end{flalign*}
so
\begin{flalign*}
K_p&=\frac{\Big(\frac{n_BRT}{V}\Big)^b}{\Big(\frac{n_ART}{V}\Big)^a}=\frac{\Big(\frac{n_B}{V}\Big)^b}{\Big(\frac{n_A}{V}\Big)^a}(RT)^{b-a}\\
  &=\frac{[B]^b}{[A]^a}RT^{b-a}=K_cRT^{b-a}
\end{flalign*}
Here the book says that $b-a= \Delta n =$ moles of gaseous products - moles of gaseous reactants.\\
\textit{\textbf{ Example 14.4} }\\
Methanol \ch{CH3OH} is manufactured industrially by the reaction
\begin{flalign*}
ch{CO}(g)+2\ch{H2}(g)\ch{<=>}\ch{CH3OH}(g)
\end{flalign*}
The equilibrium constant $K_c$ is 10.5 at $220^o$C. What is the value of $K_p$ at this temperature
\begin{flalign*}
K_p&=K_cRT^{b-a}\\
   &=10.5\cdot 0.0821\cdot(273+220)^{(1-3)}\\
   &=6.4093e-03 \approx 6.41\cdot 10^{-3}
\end{flalign*}
\hspace{1em}\\
\textit{\textbf{ Le Ch\^atelier's Principle} }\\
\textit{If an external stress is applied to a system in equilibrium, the system adjusts in such a way that the stress is partially offset as the system reaches
a new equilibrium position.}

A temperature increase favours the endothermic reaction (the reaction which absorbes heat). If left side absorbs
heat then the reaction will increase the products.\\
A temperature decrease favours the exotermic reaction (the reaction releases heat). If the left side releases heat then the reaction
will increase the products.\\
\end{multicols}



\vfill\null
\clearpage
%\columnbreak
%\newpage


%\vfill\null
%\clearpage
%\columnbreak
%\newpage



%\underbrace{}

% \hspace{1em}

%\begin{enumerate}[label=(\alph*)]
%\end{enumerate}

%$$
%  A = 
%  \begin{bmatrix}
%    1 & 0  & 2i\\
%    2i & 0 &  -4\\
%    -i &  0 & -2i\\
%  \end{bmatrix}
%$$

%\begin{flalign*}
%  A = 
%  \begin{bmatrix}
%    1 & 0  & 2i\\
%    2i & 0 &  -4\\
%    -i &  0 & -2i\\
%  \end{bmatrix}
%\end{flalign*}


%\begin{flalign*}
%\psi(x) = \begin{cases} Ae^{ikx}+Be^{-ikx} &\ \  x<-a \\
%                        Ce^{\kappa x}+De^{-\kappa x} &\ \ -a < x < a\\
%						Fe^{ikx} & \ \ x>a
%       \end{cases}
%\end{flalign*}

%\begin{figure}[H]
%  \includegraphics[width=\linewidth]{odd_finite.eps}
%  \caption{$z_0=0.1\pi,0.5\pi, 3\pi,7\pi$}
%  \label{fig4}
%\end{figure}
\end{document}

\vfill\null
\clearpage
%\columnbreak
%\newpage










                                     
                                     



