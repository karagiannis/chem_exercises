%\documentclass[twocolumn]{article}
\documentclass[./chem_exercises.tex]{subfiles}
\begin{document}

%\begin{titlepage}
%\maketitle
%\end{titlepage}



\section{Laborationsförberedelse 1}
Denna text avser att uppfylla kraven för att få kunna utföra Laboration 2 och innehåller därför
försök till svar på de obligatoriska frågor som krävs för delatagande i labben samt redovining
av den beräkningar som skall uppvisas avseende tillverkning av två buffertar enligt laborationskompmendiet.
Tilldelat labbgrupp nummer är C5. Detta innebär att buffertarna ska tillverkas enligt följande specifikaton.
\begin{flalign*}
\text{Buffert 1} &= \text{pH } 3.5\\
\text{Buffert 2} &= \text{pH } 7.4\\
\end{flalign*}

\begin{multicols}{2}
\subsection{Frågor att besvara innan laborationen}
\begin{enumerate}[label=(\alph*)]
\item \textit{Varför blir lösningen varm när \ch{NaOH} och vatten blandas?}\\

Hydratiseringsentalpin för \ch{NaOH} är -44.5 kJ/mol då volymen $\ch{H2O}\gg$ volymen $\ch{NaOH}(aq)$\footnote{\url{https://chemistry.stackexchange.com/questions/147458/enthalpy-of-dissolution-of-naoh-in-small-amount-of-water}}.
Det negativa tecknet betyder att hydratiseringen är en exoterm reaktion - energi avges.\\

\item \textit{Varför används en plastbehållare för lösningen (i stället för en glasflaska)?}\\

Därför att hydratiserad \ch{NaOH} etsar glas\footnote{\url{https://chemistry.science.narkive.com/tCRaUcEV/why-is-naoh-sodium-hydroxide-stored-in-a-plastic-bottle}}\\

\item \textit{Beräkna lösningens NaOH-koncentration uttryckt i molal (mol/kg
lösningsmedel).}\\

Molmassan $M$ för \ch{NaOH} är summan av de ingående atomernas molmassor
\begin{flalign*}
M(\ch{Na})&=22.989\\
M(\ch{O})&=15.9994\\
M(\ch{H})&=1.00794\\
M(\ch{Na})+M(\ch{O})+M(\ch{H})&=39.996\\
                              &\approx 40.0 \frac{\text{ g}}{\text{ mol}}
\end{flalign*}
$25$g \ch{NaOH} motsvarar således
\begin{flalign*}
n&=\frac{m}{M}=\frac{25\text{g}}{40\text{g}\cdot\text{ mol}^{-1}}=0.6250 \text{ mol}
\end{flalign*}
25 ml \ch{H2O} skulle användas till lösningen. Densiteten för \ch{H2O} är 1 mg per ml således är massan för vattnet 25 mg.
\begin{flalign*}
25\text{mg} &=0.025\text{g}=0.025\times 10^{-3}\text{ kg}\\
            &=2.5\times10^{-5}\text{ kg}
\end{flalign*}
\begin{flalign*}
molal &=\frac{n}{m_{\ch{NaOH}}+m_{\ch{H20}}}\\
      &=\frac{0.6250\text{mol}}{2\cdot 2.5\times10^{-5}\text{ kg}}\\
      &=12502\frac{\text{mol}}{\text{ kg}}
\end{flalign*}

\item \textit{Ungefär hur stor blir den totala ammoniakkoncentrationen i bufferten uttryckt i
M? 25 viktsprocent ammoniaklösning har en densitet  0.91 kg per liter.}\\

Svar: 0.013359 M, men läs igenom texten, snälla och ge mig feedback på den därför att mina beräkningar
visar att den här buferten inte fungerar.\\

Till 63.6 ml ammoniaklösning $\ch{NH2}(aq)$ skall 7 g ammoniumklorid $\ch{NH4Cl}(s)$ tillsättas.
Frågan avser tydligen ammoniakkoncentrationen vid jämvikt.\\

Ammoniaklösningen beskrivs av följande jämviktsförhållande
\begin{flalign*}
\ch{NH3}(aq)+\ch{H2O}(l)\ch{<=>} \ch{NH4^+}(aq)+\ch{OH^-}(aq)
\end{flalign*}
därefter ska 7 gram av saltet ammoniumklorid $\ch{NH4Cl}(s)$ tillsättas till ammoniaklösningen
\begin{flalign*}
\ch{NH4Cl}(s)\xrightarrow{\ch{H2O}} \ch{NH4^+}(aq)+\ch{Cl^-}(aq)
\end{flalign*}
\ch{NH4^+} är en svag konjugerad syra till den svaga basen \ch{NH3}
 och allt sammantaget ska sedan spädas till 100 ml.\\
 
\textbf{ 
Om $\ch{H^+}$ tillsätts i lösningen så kommer dessa att tas upp av \ch{OH^-} och bilda vatten
varför högerledet blir surare men då kommer Le Chateliers princip ``verka'' så att
vattnet i vänsterledet bildar fler \ch{OH^-}.Är detta korrekt?\\
}\\

\textbf{
Om \ch{OH^-} tillsätts i lösningen så kommer högerledet bli basiskt och vänsterledet kommer att verka
på sådant sätt att fler \ch{NH4^+} bildas. Är detta korrekt?\\
}

Vi måste dock först beräkna ursprungskoncentrationen av $\ch{NH3}(aq)$ för att kunna avgöra vilken
den totala ammoniakkoncentrationen är i den färdiga bufferten.\\


63.6 ml av 25-viktprocentslösningen har massan
\begin{flalign*}
m_{\text{lösning}} &=\rho\cdot V \\
&=0.91\frac{\text{ kg}}{\text{dm}^3}\cdot 0.0636\text{ dm}^3\\
&=0.057876\text{ kg} =57.876\text{ g}\\
\end{flalign*}
där $m_{\text{lösning}}=m_{\text{ingrediens}}+m_{\text{lösningsmedel}}$.\\

Formeln för viktprocent ger med insatta värden
\begin{flalign*}
\text{viktprocent} &=\frac{m_{\text{ingrediens}}}{m_{\text{ingrediens}}+m_{\text{lösningsmedel}}}\\
0.25&=\frac{m_{\text{ingrediens}}}{57.876}\iff\\
m_{\text{ingrediens}}&=0.25\cdot57.876=0.014469 \text{ g}
\end{flalign*}
Vi har alltså totalmassan för ingredienserna som är i ammoniaklösningen är \ch{NH3}, \ch{NH4^+} och \ch{OH^-}
men vi behöver koncentrationen i Molar för att kunna fortsätta....\\
För att beräkna koncentrationerna så behövs respektive substansmängder (som kommer att vara 1:1:1).\\
Hur räknar jag fram totala massan ingredienser? Syreatomer finns ju med nu som är upptagna ur vattnet.\\
Vi vet att
\begin{flalign*}
K_b &=\frac{[\ch{NH4^+}][\ch{OH^-}]}{[\ch{NH3}]}=1.8\times10^{-5}&(1)\\
\end{flalign*}
Vi vet också att lösningens jämvikt kan beskrivas enligt ICE-uppställningen såsom
\begin{center}
\begin{tabular}{c c c c c c} 
  & \ch{NH3} & \!\!\ch{<=>}\!\! &\ch{NH4^+}& \!\!+\!\!&\ch{OH^-} \\ 
I(M): &y \!\!\!  &  &0.00& \!\!\! &0.00\\ 
C(M):  &-x     & \!\!\!&x   & \!\!\! &x\\ 
\hline
E(M): &y-x  & \!\!\! &x& \!\!\! &x\\
\end{tabular}
\end{center}
Kan vi anta att  $m_{\text{ingrediens}}\approx m(\ch{NH3})$  och därför såsom följd att koncentrationerna 
$y$ och $x$ förhåller sig såsom $y\gg x$?\\
Vi sätter in i ekvation (1)
\begin{flalign*}
K_b &=\frac{x^2}{y-x}=1.8\times10^{-5}\\
\end{flalign*}
Vi vet att den kemiska processen natur påför kravet $y> x$, men detta villkoret
ger om man beaktar kvotens litenhet att $y\gg x$ så följande förenkling är rimlig:
\begin{flalign*}
K_b &\approx\frac{x^2}{y}=1.8\times10^{-5}\\
\end{flalign*}
vilket betyder att åtminstone $y>100\cdot x$ eller större.\\
Eftersom koncentrationen $y$ är mycket större än $x$ så ger detta 
även att \\
$m(\ch{NH3})\gg m(\ch{OH^-})> m(\ch{NH4^+})$. \\
Antar således att totala massan av de ingående delarna i hela lösningens massa approximativt är $m(\ch{NH3})$.\\
dvs. \\
\begin{flalign*}
m(\ch{NH3})&\approx m(\ch{OH^-})+m(\ch{NH4^+})+m(\ch{NH3})\\
           &=0.014469 \text{ g}
\end{flalign*}
Molmassan $M(\ch{NH3})$ är
\begin{flalign*}
M(\ch{NH3})&=14.0067+3\cdot 1.0079\\
           &=17.030 \text{ g}\cdot\text{mol}^{-1}
\end{flalign*}
Koncentrationen  $c(\ch{NH3})$ i Molar är
\begin{flalign*}
c(\ch{NH3})&=\frac{n(\ch{NH3})}{V} \\
 &=\frac{\frac{m(\ch{NH3})}{M(\ch{NH3})}}{V}=\frac{m(\ch{NH3})}{M(\ch{NH3})\cdot V}\\
 &=\frac{0.014469\text{ g}}{17.030 \text{ g}\cdot\text{mol}^{-1}\cdot0.0636\text{ dm}^3}\\
 &=0.013359 \text{ mol}\cdot\text{dm}^{-3}\\
 &=0.013359 M(olar)
\end{flalign*}
Vi tittar nu på 7 g av saltet som ska tillsättas och den resulterande konjugerande syran.
\begin{flalign*}
\ch{NH4Cl}(aq)\rightarrow \ch{NH4^+}(aq)+\ch{Cl^-}\\
\end{flalign*}
Substansmängderna och koncentrationerna då lösningen är utspädd till 100 ml
förhåller sig såsom 1:1:1.  Molmassan M(\ch{NH4Cl}) är
\begin{flalign*}
M(\ch{NH4Cl})&=14.00647+4\cdot1.00794+35.4527\\
             &=53.491 \text{ g}\cdot\text{mol}^{-1}\\
\end{flalign*}
där substansmängden $n$ ges av
\begin{flalign*}
n(\ch{NH4Cl})&=\frac{m_{\ch{NH4Cl}}}{M(\ch{NH4Cl})}\\
             &=\frac{7\text{ g}}{53.491 \text{ g}\cdot\text{mol}^{-1}}\\
             &=0.1309\text{ mol}
\end{flalign*}
Substansmängderna är lika:\\
$n(\ch{NH4Cl})=n(\ch{NH4^+})$\\

Koncentrationen $c(\ch{NH4^+})$ ges av
\begin{flalign*}
c(\ch{NH4^+})&=\frac{n(\ch{NH4^+})}{V_{tot}}\\
             &=\frac{0.1309\text{ mol}}{0.1\text{dm}^3}\\
             &=1.309 M(olar)
\end{flalign*}

Initialkoncentrationen, förändringarna och slutkoncentrationerna är\\
(Avrundar för att få plats $[\ch{NH4^+}]\approx 1.31M$ och
$[\ch{NH3}]=0.013359 M\approx 0.01M$
\begin{center}
\begin{tabular}{c c c c c c} 
  & \ch{NH3} & \!\!\ch{<=>}\!\! &\ch{NH4^+}& \!\!+\!\!&\ch{OH^-} \\ 
I(M): &0.01 \!\!\!  &  &1.31& \!\!\! &0.00\\ 
C(M):  &-x     & \!\!\!&x   & \!\!\! &x\\ 
\hline
E(M): &0.01-x  & \!\!\! &x+1.31& \!\!\! &x\\
\end{tabular}
\end{center}
\begin{flalign*}
K_b &=\frac{[\ch{NH4^+}][\ch{OH^-}]}{[\ch{NH3}]}=1.8\times10^{-5}&(1)\\
    &=\frac{(x+1.31)x}{0.01-x}=1.8\times10^{-5}&(1)\\
\end{flalign*}
Andragradaren blir
\begin{flalign*}
\frac{(x+1.31)x}{0.01-x}&=1.8\times10^{-5}\iff\\
x^2+1.31&=(0.01-x)1.8\times10^{-5}\\
x^2+(1.31&+1.8\times10^{-5})-1.8\cdot10^{-7}=0\\
\end{flalign*}
Octave
\begin{verbatim}
>> roots([1,(1.31+1.8E-5),-1.8E-7])
ans =

  -1.3100e+00
   1.3740e-07
\end{verbatim}

Koncentrationer kan inte vara negativa så därför förskastas första roten.
Behåller $x_2= 1.3740e-07$ men denna är i prncip noll så något är fel.
Den totala ammoniakkoncentrationen [\ch{NH3}]
\begin{flalign*}
[\ch{NH3}]&=0.013359 -x_2\\
          &=0.013359 -1.3740e-07\\
          &=0.013359
\end{flalign*}
Svar med tre värde siffror $[\ch{NH3}]=0.0134M$
Vid jämvikt har vi dessutom
Koncentrationen 
\begin{flalign*}
[\ch{OH^-}]&= x_2=1.3740e-07\\
pOH &= -log(1.3740e-07)\\
    &=6.8620\\
pH &=14.00-6.8620\\
   &=7.1380
\end{flalign*}
vilket ligger utanför bufferspannet så någonting måste vara fel.\\
Enligt Youtube\footnote{\url{https://www.youtube.com/watch?v=zdP8hKOnukc}}
så skall man göra approximationerna
\begin{flalign*}
[\ch{NH3}]&\approx \text{ molarity of base}\\
[\ch{NH4^+}]&\approx \text{ molarity of salt}\\
\end{flalign*}
och man behöver tydligen inte göra en jämviktsberäkning. Varför?\\
\begin{flalign*}
K_b &=\frac{[\ch{NH4^+}][\ch{OH^-}]}{[\ch{NH3}]}=1.8\times10^{-5}\\
[\ch{OH^-}]&=\frac{[\ch{NH3}]K_b}{[\ch{NH4^+}]}\\
           &=\frac{0.013359\cdot 1.8\times10^{-5}}{1.309}\\
           &=1.8370e-07
\end{flalign*}
där
\begin{flalign*}
pOH &=-log(1.8370e-07)\\
    &=6.7359\\
pH &=14-6.7359\\
   &=7.2641
\end{flalign*}
samma skit. Något är fel.

\item \textit{Räkna ut vilket teoretiskt pH-värden bufferten borde få.}\\

Avseende ammoniakbufferten till Lab3 så får jag pH till 7.2641 vilket inte stämmer med buffertspannet.\\

\item \textit{Vad kännetecknar en bra syra/bas-buffert! Motivera!}\\
Den kan hålla pH värdet konstant vid tillsättning av syror eller baser.

\item \textit{Skicka in recept på hur buffertarna ska göras till din handledare.}\\

\item \textit{Beskriv kortfattat med reaktionsformler vad som händer när den buffert du valt
blir utsatt för en störning i form av stark protolyt (välj fritt typ av stark protolyt.)}

\item \textit{Bufferten måste justeras till korrekt pH-värde efter att den gjorts enligt recept.
Varför kan det behövas trots att ni gjort bufferten noga?}\\
Därför att approximationer som gjorts i räkningarna ofta är för grova.

\end{enumerate}
\end{multicols}

\section{Buffert 1 pH 3.5}
\begin{multicols}{2}
Enligt utdelade pappret ``Buffertspann'' så är det lämpligaste syra/konjugerade baspar
Myrsyra/Formiatjonen
\begin{flalign*}
\ch{HCOOH}(aq)\ch{<=>}\ch{H^+}(aq) + \ch{HCOO^-}(aq)
\end{flalign*}
där
\begin{flalign*}
K_a&=1.7\times 10^{-4}\\
pK_a&=-log(1.7\times 10^{-4})=3.7696
\end{flalign*}
Vi ska åstadkomma pH 3.5 vilket insättes i vänsterledet
\begin{flalign*}
pH &=pK_a + log\frac{[\ch{HCOO^-}]}{[\ch{HCOOH}]}\\
3.5&=3.7696+ log\frac{[\ch{HCOO^-}]}{[\ch{HCOOH}]}\iff\\
log\frac{[\ch{HCOO^-}]}{[\ch{HCOOH}]}&=-0.2696
\end{flalign*}
eller om vi tar antilogaritmen
\begin{flalign*}
\frac{[\ch{HCOO^-}]}{[\ch{HCOOH}]}&=10^{-0.2696}\\
                              &=0.5375\\
							  &=\frac{1}{1.8605}\\
\end{flalign*}
Vilket betyder att det molära förhållandet
\begin{flalign*}
\frac{n(\ch{HCOONa})}{n(\ch{HCOOH})}&=\frac{1}{1.8605}\\
\end{flalign*}
Molmassorna för respektive förening är
\begin{flalign*}
M(\ch{HCOONa})&=1.01+12.01+2\cdot 16.00+22.99\\
              &=68.010\text{ g}\cdot\text{mol}^{-1}\\\\
M(\ch{HCOOH})&=&=2\cdot1.01+12.01+2\cdot 16.00\\ 
               &=46.030 \text{ g}\cdot\text{mol}^{-1}\\
\end{flalign*}
För att bestämma den erfoderliga substansmängden så har vi formeln
\begin{flalign*}
c&=\frac{n}{V}\iff\\
n(\ch{HCOONa})&=c\cdot V=0.1\cdot 0.1 = 0.01\\
n(\ch{HCOOH})&=1.8\cdot n(\ch{NaH2PO4})=0.018\\
\end{flalign*}
Massan m(\ch{HCOONa}) som ska vägas upp är
\begin{flalign*}
m(\ch{HCOONa})&=n(\ch{HCOONa})\cdot M(\ch{HCOONa})\\
               &=0.01\cdot 68.010\\
			   &=0.68 \text{g}\\
\end{flalign*}
Massan m(\ch{HCOOH}) som ska vägas upp är
\begin{flalign*}
m(\ch{HCOOH})&=n(\ch{HCOOH})\cdot M(\ch{HCOOH})\\
               &=0.018\cdot 46.030\\
			   &=0.8285 \text{g}\\
\end{flalign*}
men myrsyran är i flytande form. Antar för räkningens skull att den finns tillgänglig specificerad med
en viktprocent $v_{\%}$
\begin{flalign*}
v_{\%}&=\frac{m(\ch{HCOONa})}{m_{sol}}\iff\\
m_{sol}&=\frac{m(\ch{HCOONa})}{v_{\%}}
\end{flalign*}
Massorna blandas och lösning späds till 100ml.

\end{multicols}
\section{Buffert 2 pH 7.4}
\begin{multicols}{2}
Enligt utdelade pappret ``Buffertspann'' så är det lämpligaste syra/konjugerade baspar
\ch{H2PO4^-}/\ch{HPO4^{2-}}.\\
Kursboken har en fullständig lösning och anger att molförhållandet 1.5:1 ska gälla
\begin{flalign*}
\frac{[\ch{HPO4^{2-}]}}{[\ch{H2PO4^-}]}&=1.5\\
\end{flalign*}
Man skall lösa 1.5 mol \ch{Na2HPO4} och 1.0 mol \ch{NaH2PO4} i vatten.
Buffertkoncentrationen skall vara 0.1 M och volymen skall vara 100 ml.\\

Molmassorna för respektive förening är
\begin{flalign*}
M(\ch{Na2HPO4})&= 2\cdot22.99+1.01+30.97+4\cdot 16.00\\
               &=141.96 \text{ g}\cdot\text{mol}^{-1}\\
M(\ch{NaH2PO4})&= 22.99+2\cdot1.01+30.97+4\cdot 16.00\\
               &=119.98 \text{ g}\cdot\text{mol}^{-1}\\
\end{flalign*}
För att bestämma den erfoderliga substansmängden så har vi formeln
\begin{flalign*}
c&=\frac{n}{V}\iff\\
n(\ch{NaH2PO4})&=c\cdot V=0.1\cdot 0.1 = 0.01\\
n(\ch{Na2HPO4})&=1.5\cdot n(\ch{NaH2PO4})=0.015\\
\end{flalign*}

Massorna som ska vägas upp är
\begin{flalign*}
m(\ch{NaH2PO4})&=n(\ch{NaH2PO4})\cdot M(\ch{NaH2PO4})\\
               &=0.01\cdot 119.98\\
			   &=1.1998 \text{g}\\
\end{flalign*}
samt
\begin{flalign*}
m(\ch{Na2HPO4})&=n(\ch{Na2HPO4})\cdot M(\ch{Na2HPO4})\\
               &=0.015\cdot 141.96\\
			   &=2.1294\text{g}\\
\end{flalign*}


\end{multicols}
\section{Mätverktyg}
Precisionspipett skall användas.


%\vfill\null
%\clearpage
%\columnbreak
%\newpage



%\underbrace{}

% \hspace{1em}

%\begin{enumerate}[label=(\alph*)]
%\end{enumerate}

%$$
%  A = 
%  \begin{bmatrix}
%    1 & 0  & 2i\\
%    2i & 0 &  -4\\
%    -i &  0 & -2i\\
%  \end{bmatrix}
%$$

%\begin{flalign*}
%  A = 
%  \begin{bmatrix}
%    1 & 0  & 2i\\
%    2i & 0 &  -4\\
%    -i &  0 & -2i\\
%  \end{bmatrix}
%\end{flalign*}


%\begin{flalign*}
%\psi(x) = \begin{cases} Ae^{ikx}+Be^{-ikx} &\ \  x<-a \\
%                        Ce^{\kappa x}+De^{-\kappa x} &\ \ -a < x < a\\
%						Fe^{ikx} & \ \ x>a
%       \end{cases}
%\end{flalign*}

%\begin{figure}[H]
%  \includegraphics[width=\linewidth]{odd_finite.eps}
%  \caption{$z_0=0.1\pi,0.5\pi, 3\pi,7\pi$}
%  \label{fig4}
%\end{figure}
\end{document}






\vfill\null
\clearpage
\columnbreak
\newpage










                                     
                                     



