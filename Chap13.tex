%\documentclass[twocolumn]{article}
\documentclass[./chem_exercises.tex]{subfiles}
\begin{document}
\begin{multicols}{2}
%\begin{titlepage}
%\maketitle
%\end{titlepage}
\textit{\textbf{Chapter 13 Chemical kinetics - summary} }\\

\textit{\textbf{ 13.1 The rate of a reaction summary} }\\
Chemical kinetics is the area of chemistry concerned with the speeds, or
rates at which a chemical reaction occurs.
\begin{flalign*}
\ch{Br2}(aq)&+\ch{HCOOH}(aq)\rightarrow \\
&2\ch{Br}^-(aq)+2\ch{H}^+(aq) + \ch{CO2}(g)
\end{flalign*}
The average rate is 
rates at which a chemical reaction occurs.
\begin{flalign*}
\text{average rate} = -\frac{\Delta[\ch{Br2}]}{\Delta t}\\
\end{flalign*}
Let $\Delta t$ go to zero and define the ``rate''. It can be shown that the quotinent
between the time-varying instantaneous rate and corresponding
concentration is a constant.
\begin{flalign*}
k=\frac{\text{rate}}{[\ch{Br2}]}
\end{flalign*}
In general for the reaction
\begin{flalign*}
a\ch{A}+b\ch{B}\rightarrow c\ch{C}+d\ch{D}
\end{flalign*}
the rate is given by
\begin{flalign*}
\text{rate}&=-\frac{1}{a}\frac{\Delta[\ch{A}]}{\Delta t}=-\frac{1}{b}\frac{\Delta[\ch{B}]}{\Delta t}\\
            &=\frac{1}{c}\frac{\Delta[\ch{C}]}{\Delta t}=\frac{1}{d}\frac{\Delta[\ch{D}]}{\Delta t}\\
\end{flalign*}
\textit{\textbf{ 13.2 Rate laws - summary} }\\
The rate law expresses the relationship of the rate of the reaction and the concentrations
of the reactants raised to some powers.\\
Idiot definition.\\
If 
\begin{flalign*}
a\ch{A}+b\ch{B}\rightarrow c\ch{C}+d\ch{D}
\end{flalign*}
then
\begin{flalign*}
\text{rate} &= k[A]^x[B]^y\iff\\
          k&=\frac{\text{rate}}{[A]^x[B]^y}
\end{flalign*}

\textit{\textbf{ 13.3 The relation between reactant concentration and time - summary} }\\

\textit{\textbf{First-order reactions} }\\ 
A first-order reaction is a reaction whose rate depends on the reactant concentration raised to the first power.
In a first order reation of type
\begin{flalign*}
\ch{A}\rightarrow \text{product}
\end{flalign*}
we have
\begin{flalign*}
\text{rate}&=-\frac{\Delta[\ch{A}]}{\Delta t}\\
\end{flalign*}
From the rate law we also know that
\begin{flalign*}
k&=\frac{\text{rate}}{[A]}\iff\\
\text{rate}&=k[A]
\end{flalign*}
Putting these together yields
\begin{flalign*}
k&=\frac{\text{rate}}{[A]}\iff\\
-\frac{\Delta[\ch{A}]}{\Delta t}&=k[A]
\end{flalign*}
Solving it gives
\begin{flalign*}
ln[A]_t = -kt+ln[A]_0
\end{flalign*}

\textit{\textbf{Second-order reactions} }\\ 
A second order reaction is a reaction whose rate depends on the concentration
of one reactant raised to the second power or on the concentrations
of two different reactnants, each raised to the first power

\textit{\textbf{The Arrhenius Equation} }\\
The dependence of the rate constant of a reaction on temperature can be
expressed by the Arrhenius equation
\begin{flalign*}
k&=Ae^{-E_a/RT}
\end{flalign*}
\end{multicols}
\vfill\null
\clearpage
%\columnbreak
%\newpage

%\vfill\null
%\clearpage
%\columnbreak
%\newpage



%\underbrace{}

% \hspace{1em}

%\begin{enumerate}[label=(\alph*)]
%\end{enumerate}

%$$
%  A = 
%  \begin{bmatrix}
%    1 & 0  & 2i\\
%    2i & 0 &  -4\\
%    -i &  0 & -2i\\
%  \end{bmatrix}
%$$

%\begin{flalign*}
%  A = 
%  \begin{bmatrix}
%    1 & 0  & 2i\\
%    2i & 0 &  -4\\
%    -i &  0 & -2i\\
%  \end{bmatrix}
%\end{flalign*}


%\begin{flalign*}
%\psi(x) = \begin{cases} Ae^{ikx}+Be^{-ikx} &\ \  x<-a \\
%                        Ce^{\kappa x}+De^{-\kappa x} &\ \ -a < x < a\\
%						Fe^{ikx} & \ \ x>a
%       \end{cases}
%\end{flalign*}

%\begin{figure}[H]
%  \includegraphics[width=\linewidth]{odd_finite.eps}
%  \caption{$z_0=0.1\pi,0.5\pi, 3\pi,7\pi$}
%  \label{fig4}
%\end{figure}
\end{document}

\vfill\null
\clearpage
\columnbreak
\newpage










                                     
                                     



