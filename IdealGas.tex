%\documentclass[twocolumn]{article}
\documentclass[./chem_exercises.tex]{subfiles}
\begin{document}
\begin{multicols}{2}
%\begin{titlepage}
%\maketitle
%\end{titlepage}
\textit{\textbf{Ideal Gas Equation} }\\
\begin{flalign*}
R&=\frac{pV}{nT}=\frac{101.325\cdot 10^3 \cdot 0.022414}{1\cdot 273.15}\\
  &=8.314473915 \frac{\text{J}}{\text{mol}\cdot\text{K}}
\end{flalign*}
If we want to use pressure in atmosferes
\begin{flalign*}
pV &=nRT\\
p'\cdot101.325\cdot 10^3\cdot V &=nRT\\
p'V&=\frac{nRT}{101.325\cdot 10^3}\\
R'&=\frac{8.314473915}{101.325\cdot 10^3}
\end{flalign*}
If we also want to use liters
\begin{flalign*}
p'V&=nR'T\\
p'\frac{V'}{1000}&=nR'T\\
p'V'&=nR'T\cdot 1000\\
R''&=\frac{8.314473915\cdot 1000}{101.325\cdot 10^3}\\
   &=0.082057 \frac{\text{L}\cdot \text{atm}}{\text{mol}\cdot\text{K}}
\end{flalign*}
If we have 1.82 moles of gas in a vessel of volume 5.43l at $69.5^o$C the pressure in atmosferes
is
\begin{flalign*}
p'V'&=nR''T\iff\\
p'&=\frac{nR''T}{V'}=\frac{1.82\cdot 0.082057\cdot (273+69.5)}{5.43}\\
  &=9.4199 \text{ atm}
\end{flalign*}
\end{multicols}
\vfill\null
\clearpage
%\columnbreak
%\newpage

%\vfill\null
%\clearpage
%\columnbreak
%\newpage



%\underbrace{}

% \hspace{1em}

%\begin{enumerate}[label=(\alph*)]
%\end{enumerate}

%$$
%  A = 
%  \begin{bmatrix}
%    1 & 0  & 2i\\
%    2i & 0 &  -4\\
%    -i &  0 & -2i\\
%  \end{bmatrix}
%$$

%\begin{flalign*}
%  A = 
%  \begin{bmatrix}
%    1 & 0  & 2i\\
%    2i & 0 &  -4\\
%    -i &  0 & -2i\\
%  \end{bmatrix}
%\end{flalign*}


%\begin{flalign*}
%\psi(x) = \begin{cases} Ae^{ikx}+Be^{-ikx} &\ \  x<-a \\
%                        Ce^{\kappa x}+De^{-\kappa x} &\ \ -a < x < a\\
%						Fe^{ikx} & \ \ x>a
%       \end{cases}
%\end{flalign*}

%\begin{figure}[H]
%  \includegraphics[width=\linewidth]{odd_finite.eps}
%  \caption{$z_0=0.1\pi,0.5\pi, 3\pi,7\pi$}
%  \label{fig4}
%\end{figure}
\end{document}

\vfill\null
\clearpage
\columnbreak
\newpage










                                     
                                     



