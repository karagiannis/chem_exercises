%\documentclass[twocolumn]{article}
\documentclass[./chem_exercises.tex]{subfiles}
\begin{document}

%\begin{titlepage}
%\maketitle
%\end{titlepage}



\section{Laborationsförberedelse 1}
Denna text avser att uppfylla kraven för att få kunna utföra Laboration 2 och innehåller därför
försök till svar på de obligatoriska frågor som krävs för delatagande i labben samt redovining
av den beräkningar som skall uppvisas avseende tillverkning av två buffertar enligt laborationskompmendiet.
Tilldelat labbgrupp nummer är C5. Detta innebär att buffertarna ska tillverkas enligt följande specifikaton.
\begin{flalign*}
\text{Buffert 1} &= \text{pH } 3.5\\
\text{Buffert 2} &= \text{pH } 7.4\\
\end{flalign*}


\subsection{Frågor att besvara innan laborationen}
\begin{enumerate}[label=(\alph*)]
\item \textit{Varför blir lösningen varm när \ch{NaOH} och vatten blandas?}\\

Hydratiseringsentalpin för \ch{NaOH} är -44.5 kJ/mol då volymen $\ch{H2O}\gg$ volymen $\ch{NaOH}(aq)$\footnote{\url{https://chemistry.stackexchange.com/questions/147458/enthalpy-of-dissolution-of-naoh-in-small-amount-of-water}}.
Det negativa tecknet betyder att hydratiseringen är en exoterm reaktion - energi avges.\\

\item \textit{Varför används en plastbehållare för lösningen (i stället för en glasflaska)?}\\

Därför att hydratiserad \ch{NaOH} etsar glas\footnote{\url{https://chemistry.science.narkive.com/tCRaUcEV/why-is-naoh-sodium-hydroxide-stored-in-a-plastic-bottle}}\\

\item \textit{Beräkna lösningens NaOH-koncentration uttryckt i molal (mol/kg
lösningsmedel).}\\

Molmassan $M$ för \ch{NaOH} är summan av de ingående atomernas molmassor
\begin{flalign*}
M(\ch{Na})&=22.989\\
M(\ch{O})&=15.9994\\
M(\ch{H})&=1.00794\\
M(\ch{Na})+M(\ch{O})+M(\ch{H})&=39.996\\
                              &\approx 40.0 \frac{\text{ g}}{\text{ mol}}
\end{flalign*}
$25$g \ch{NaOH} motsvarar således
\begin{flalign*}
n&=\frac{m}{M}=\frac{25\text{g}}{40\text{g}\cdot\text{ mol}^{-1}}=0.6250 \text{ mol}
\end{flalign*}
25 ml \ch{H2O} skulle användas till lösningen. Densiteten för \ch{H2O} är 1 g per ml således är massan för vattnet 25g.
\begin{flalign*}
25\text{g} &=0.025\text{ kg}\\
\end{flalign*}
Därför blir molaliteten
\begin{flalign*}
     &\frac{n}{m_{\ch{H20}}(\text{lösningsmedel kg})}\\
      &=\frac{0.6250\text{mol}}{0.025\text{ kg}\text{ kg}}\\
      &=25 \text{mol per kg lösningsmedel }\ch{H2O}
\end{flalign*}

\item \textit{Ungefär hur stor blir den totala ammoniakkoncentrationen i bufferten uttryckt i
M? 25 viktsprocent ammoniaklösning har en densitet  0.91 kg per liter.}\\

Vad menas med den totala ammoniak-koncentrationen? Avser detta \ch{NH3} enbart eller skall motsvarande
vikt av \ch{NH4^+} dras ifrån från jämnviktsbalansen?
% Vi lägger därefter till 0.130 mol \ch{NH4^+} genom att tillsätta saltet och då inträffar
%ett nytt jämviktsläge för \ch{NH3}(aq) är det detta nya jämviktsläge som är \textit{den totala ammoniak koncentrationen}?
Antar  att 63.6 ml 25 vikts-\% ammoniaklösning är ett initialt värde innan någon
jämvikt infunnit sig.(Tycker att det borde formulerats tydligare.)\\

Antar att den molariteten som 63.6 ml 25 vikts-\% ammoniaklösning motsvarar är ett initialt
värde dvs. innan någon reaktionsjämvikt inställt sig.

Densiteten för ammoniaklösningen är $\rho=0.91\text{ kg}/\text{dm}^3$
\begin{flalign*}
    \rho &=0.91\text{kg}/\text{dm}^3=0.91\text{ g}/\text(mL)\\
\end{flalign*}
Massan ammoniak lösning $m_{\ch{NH3}(aq)}$ är
\begin{flalign*}
   m_{\ch{NH3}(aq)}&=\rho V\\
                   &=0.91\frac{\text{ g}}{\text{ mL}}\cdot 63.6\text{mL}\\
                   &=57.876\text{ g}
\end{flalign*}
Ur viktprocentenes definition fås massan \ch{NH3} $m_{\ch{NH3}}$
\begin{flalign*}
\text{vikt\%}&=\frac{m_{\ch{NH3}}}{m_{\ch{NH3}(aq)}}\\
        0.25&=\frac{m_{\ch{NH3}}}{57.876}\iff\\
		m_{\ch{NH3}}&=0.25\cdot 57.876 \\
		            &=14.469\text{g}
\end{flalign*}
För att beräkna molariteten behövs molmassan för \ch{NH3} $M_{\ch{NH3}}$
\begin{flalign*}
M_{\ch{NH3}}&=17.0304 \frac{\text{ g}}{\text{mol}}
\end{flalign*}
Substansmängden $n_{\ch{NH3}}$ blir därför
\begin{flalign*}
n_{\ch{NH3}} &=\frac{m_{\ch{NH3}}}{M_{\ch{NH3}}}\\
             &=\frac{14.469\text{g}}{17.0304\text{ g}\cdot\text{mol}^{-1}}\\
			 &=0.849598365\text{ mol}
\end{flalign*}
Molariteten $c_{\ch{NH3}}$ blir därför
\begin{flalign*}
c_{\ch{NH3}}&=\frac{n_{\ch{NH3}}}{V}\\
            &=\frac{0.849598365\text{ mol}}{0.0636\text{ L}}\\
			&=13.358464858\text{ M}
\end{flalign*}
Initialt, innan följande jämviktsreaktion
\begin{flalign*}
\underset{b}{\ch{NH3}(aq)} + \underset{s}{\ch{H2O}(l)}\ch{<=>}\underset{s}{\ch{NH4^+}} + \underset{b}{\ch{OH^-}}
\end{flalign*}
så är molariteten $13.358464858\text{M}$ för \ch{NH3}(aq).
ICE-tabell ger följande där vattnets autoprotolys försummas.
\begin{center}
\hspace{1em}\\
\begin{tabular}{c c c c c c} 
  & \ch{NH3} & \!\!\ch{<=>}\!\! &\ch{NH4^+}& \!\!+\!\!&\ch{OH^-} \\ 
I(M): &13.36 \!\!\!  &  &0.00& \!\!\! &0.00\\ 
C(M):  &-x     & \!\!\!&x   & \!\!\! &x\\ 
\hline
E(M): &13.36-x  & \!\!\! &x& \!\!\! &x\\
\end{tabular}
\end{center}

Basjoniseringskonstant $K_b$ är enligt boken $1.8\times 10^{-5}\text{M}$
Således fås sambandet ur ekvationen för $K_b$ enligt
\begin{flalign*}
K_b&=\frac{[\ch{NH4^+}][\ch{OH^-}]}{[\ch{NH3}]}\\
1.8\times 10^{-5}&=\frac{x^2}{13.358-x}\\
\end{flalign*}
Antar att $13.36\gg x$
\begin{flalign*}
1.8\times 10^{-5}&=\frac{x^2}{13.358}\iff\\
x&=\pm\sqrt{13.358\cdot 1.8\times 10^{-5} }\\
 &=\pm 0.015506515
\end{flalign*}
Joniseringsfraktionen är
\begin{flalign*}
\text{Joniseringsfraktion}&=\frac{0.015506515}{13.358464858}\\
&=1.1608e-03
\end{flalign*}
vilket visar att antagandet var rimligt.
Löser andragradaren för fullständighetens skull
i Matlab.
\begin{verbatim}
%Kb=x^2/(a-x);
clear all
Kb=1.8E-5
a=13.358464858
%x^2+K*x-Kb=0
c=[1,Kb,-Kb*a];
sol = roots(c)
w=1E-14;
if(sol(1)>0)
  disp('sol(1)>0')
  OH=sol(1)
  pOH=-log10(OH)
  H=w/OH
else
  if(sol(2)>0)
    disp('sol(2)>0')
     OH=sol(2)
     pOH=-log10(OH)
     H=w/OH
  end
end
pH=-log10(H)
pH_from_pOH=14-pOH
\end{verbatim}
Vilket ger utskriften
\begin{verbatim}
Kb = 1.8000e-05
a = 13.358
sol =

  -0.015516
   0.015498

sol(2)>0
OH = 0.015498
pOH = 1.8097
H = 6.4526e-13
pH = 12.190
pH_from_pOH = 12.190
\end{verbatim}
Vid jämvikt är koncentrationerna följande:
\begin{flalign*}
[\ch{NH4^+}]&=0.015498\text{ M}\\
[\ch{OH^-}]&=0.015498\text{ M}\\
[\ch{NH3}]&=13.358-0.015498=13.343\text{ M}\\
[H^+]&=6.4526\times 10^{-13}\text{ M}\\
\end{flalign*}

Nu skall 7 gram av saltet ammoniumklorid \ch{NH4Cl} tillsättas
vilket har molmassan $M_{\ch{NH4Cl}}$
\begin{flalign*}
M_{\ch{NH4Cl}}&=1.00647+4\cdot1.00794+35.4527\\
              &=53.49093\text{ g}\cdot\text{ mol}^{-1}
\end{flalign*}
Substansmängden $n_{\ch{NH4Cl}}$ är
\begin{flalign*}
n_{\ch{NH4Cl}}&=\frac{m_{\ch{NH4Cl}}\text{ g}}{M_{\ch{NH4Cl}}\text{ g}\cdot\text{ mol}^{-1}}\\
              &=\frac{7}{53.49093}\\
              &=0.130863307\text{ mol}\\
\end{flalign*}
\ch{NH4Cl} löses i vatten enligt reaktionen
\begin{flalign*}
\ch{NH4Cl}(s)+\ch{H2O}(l)\rightarrow \ch{NH4^+}(aq)+\ch{Cl^-}
\end{flalign*}
\ch{Cl^-} jonen har ingen affinitet för \ch{H^+} och ingen tendens
att hydratiseras så den är endast åskådarjon. Den lösta \ch{NH4^+}(aq)
har samma substansmängd såsom \ch{NH4Cl}(s) p.g.a. 1:1 förhållande och koncentrationen innan jämviktsreaktionen påbörjas såsom
\begin{flalign*}
c'_{\ch{NH4^+}}&=\frac{n_{\ch{NH4^+}}}{V}\\
              &=\frac{0.130863307\text{ mol}}{0.0636\text{ L}}\\
              &=2.0576\text{M}
\end{flalign*}
Ny koncentration $c''$ direkt efter tillsättning men innan jämviktsreaktion påbörjas är
\begin{flalign*}
c''_{\ch{NH4^+}}&=c'_{\ch{NH4^+}}+c_{\ch{NH4^+}}\\
                &=2.0576\text{ M}+0.015498\text{ M}\\
                &=2.0731\text{ M}
\end{flalign*}
\begin{center} 
 \hspace{1em}\\
\begin{tabular}{c c c c c c} 
  & \ch{NH3} & \!\!\ch{<=>}\!\! &\ch{NH4^+}& \!\!+\!\!&\ch{OH^-} \\ 
I(M): &13.343 \!\!\!  &  &2.07& \!\!\! &0.015\\ 
C(M):  &-x     & \!\!\!&+x   & \!\!\! &+x\\ 
\hline
E(M): &13.34-x  & \!\!\! &x+2.07& \!\!\! &x+0.015\\
\end{tabular} 
\end{center}
Ekvationen för basjoniseringskonstanten ger

\begin{flalign*}
K_b&=\frac{[\ch{NH4^+}][\ch{OH^-}]}{[\ch{NH3}]}\\
1.8\times 10^{-5}&=\frac{(x+2.07)(x+0.015)}{13.34-x}\\
\end{flalign*}
Utskriften från Matlab ger följande
\begin{verbatim}
OH = 1.1499e-04
pOH = 3.9394
H = 8.6967e-11
pH = 10.061
NH3_konc = 13.343
\end{verbatim}
Vi ser att tillsatsen av \ch{NH4^+} försköt reaktionen åt vänster
eftersom pH gick från 12.190 till 10.061.\\

Lösningen späds därefter till 100 ml.
De nya koncentrationerna blir
\begin{flalign*}
c_{100\text{ml}} &=\frac{n}{V_{new}}\\
                 &=\frac{c_{old}\cdot V_{old}}{V_{new}}\\
c_{\ch{NH3}(aq-100\text{ ml})}&=\frac{13.343\cdot 0.0636}{0.1}\\
                             &=8.486148\\
c_{\ch{H^+}(aq-100\text{ ml})}&=\frac{8.6967\times 10^{-11}\cdot 0.0636}{0.1}\\
                             &=5.5311\times10^{-11}\\
\end{flalign*}

\item \textit{Räkna ut vilket teoretiskt pH-värden bufferten borde få.}\\
\begin{flalign*}
pH&=-log10(5.5311\times10^{-11})\\
  &=10.257
\end{flalign*}


\item \textit{Vad kännetecknar en bra syra/bas-buffert! Motivera!}\\

Den kan hålla pH värdet konstant vid tillsättning av syror eller baser.

\item \textit{Skicka in recept på hur buffertarna ska göras till din handledare.}\\

\item \textit{Beskriv kortfattat med reaktionsformler vad som händer när den buffert du valt
blir utsatt för en störning i form av stark protolyt (välj fritt typ av stark protolyt.)}

Följande buffrar tilldelades
\begin{flalign*}
&\ch{HCOOH}+\ch{H2O}\ch{<=>}\ch{H^+} + \ch{HCOO^-}\\
&\ch{HCOONa}(s) +\ch{H2O}\rightarrow \ch{HCOO^-}(aq)\\
\end{flalign*}
Om \ch{H^+} tillsättes så kommer den konjugerade basen i högerledet
att omvandla oxoniumjonen till vatten
\begin{flalign*}
\underset{s}{\ch{H^+}} +\underset{b}{\ch{HCOO^-}}\rightarrow \underset{s}{\ch{HCOOH}}+\underset{b}{\ch{H2O}}\\
\end{flalign*}
Om \ch{OH^-} tillsättes så kommer den att neutraliseras av den svaga syran
i vänsterledet
\begin{flalign*}
\underset{s}{\ch{HCOOH}}+\underset{b}{\ch{OH^-}}\rightarrow \underset{b}{\ch{HCOO^-}}+\underset{s}{\ch{H2O}}\\
\end{flalign*}

Analogt för den andra bufferten
\begin{flalign*}
&\ch{H2PO4^-}+\ch{H20}\ch{<=>}\ch{H^+} + \ch{HPO4^{2-}}\\
&\ch{Na2HPO4}(s) +\ch{H2O}\rightarrow \ch{HPO4^{2-}}(aq)\\
\end{flalign*}
Om \ch{H^+} tillsättes så kommer den konjugerade basen i högerledet
att omvandla oxoniumjonen till vatten
\begin{flalign*}
\underset{s}{\ch{H^+}} +\underset{b}{\ch{HPO4^{2-}}}\rightarrow \underset{s}{\ch{H2PO4^-}}+\underset{b}{\ch{H2O}}\\
\end{flalign*}
Om \ch{OH^-} tillsättes så kommer den att neutraliseras av den svaga syran
i vänsterledet
\begin{flalign*}
\underset{s}{\ch{H2PO4^-}}+\underset{b}{\ch{OH^-}}\rightarrow \underset{b}{\ch{HPO4^{2-}}}+\underset{s}{\ch{H2O}}\\
\end{flalign*}




\item \textit{Bufferten måste justeras till korrekt pH-värde efter att den gjorts enligt recept.
Varför kan det behövas trots att ni gjort bufferten noga?}\\

Därför att vi har antagit ideala syror vilka inte påverkas av jon-par formationer och andra typer
av intramolekylära reaktioner. Motsvarigheten är allmänna gaslagen där antagandet är idela gaser.
Den effektiva koncentrationen av \ch{H^+} är ej skiljer sig lite från den ideala modellen.
Oxoniumjonen har inte riktigt samma ``aktivitet''

\end{enumerate}


\section{Buffert 1 pH 3.5}

Enligt utdelade pappret ``Buffertspann'' så är det lämpligaste syra/konjugerade baspar
Myrsyra/Formiatjonen
\begin{flalign*}
\ch{HCOOH}(aq)\ch{<=>}\ch{H^+}(aq) + \ch{HCOO^-}(aq)\\
\end{flalign*}
där
\begin{flalign*}
K_a&=1.7\times 10^{-4}\\
pK_a&=-log(1.7\times 10^{-4})=3.7696
\end{flalign*}
Vi ska åstadkomma pH 3.5 vilket insättes i vänsterledet
\begin{flalign*}
pH &=pK_a + log\frac{[\ch{HCOO^-}]}{[\ch{HCOOH}]}\\
3.5&=3.7696+ log\frac{[\ch{HCOO^-}]}{[\ch{HCOOH}]}\iff\\
log\frac{[\ch{HCOO^-}]}{[\ch{HCOOH}]}&=-0.2696\\
\end{flalign*}
eller om vi tar antilogaritmen
\begin{flalign*}
\frac{[\ch{HCOO^-}]}{[\ch{HCOOH}]}&=10^{-0.2696}\\
                              &=0.5375\\
							  &=\frac{1}{1.8605}\\
\end{flalign*}
Vilket betyder att det molära förhållandet
\begin{flalign*}
\frac{n(\ch{HCOONa})}{n(\ch{HCOOH})}&=\frac{1}{1.8605}\\
\end{flalign*}
Molmassorna för respektive förening är
\begin{flalign*}
M(\ch{HCOONa})&=1.01+12.01+2\cdot 16.00+22.99\\
              &=68.010\text{ g}\cdot\text{ mol}^{-1}\\\\
M(\ch{HCOOH})&=2\cdot1.01+12.01+2\cdot 16.00\\ 
               &=46.030 \text{ g}\cdot\text{ mol}^{-1}\\
\end{flalign*}
För att bestämma den erfoderliga substansmängden så har vi formeln
\begin{flalign*}
c&=\frac{n}{V}\iff\\
n(\ch{HCOONa})&=c\cdot V=0.1\cdot 0.1 = 0.01\\
n(\ch{HCOOH})&=1.8\cdot n(\ch{NaH2PO4})=0.018\\
\end{flalign*}
Massan m(\ch{HCOONa}) som ska vägas upp är
\begin{flalign*}
m(\ch{HCOONa})&=n(\ch{HCOONa})\cdot M(\ch{HCOONa})\\
               &=0.01\cdot 68.010\\
			   &=0.68 \text{ g}\\
\end{flalign*}
Massan m(\ch{HCOOH}) som ska vägas upp är
\begin{flalign*}
m(\ch{HCOOH})&=n(\ch{HCOOH})\cdot M(\ch{HCOOH})\\
               &=0.018\cdot 46.030\\
			   &=0.8285 \text{ g}\\
\end{flalign*}
men myrsyran är i flytande form. Antar för räkningens skull att den finns tillgänglig specificerad med
en viktprocent $v_{\%}$
\begin{flalign*}
v_{\%}&=\frac{m(\ch{HCOONa})}{m_{sol}}\iff\\
m_{sol}&=\frac{m(\ch{HCOONa})}{v_{\%}}\\
\end{flalign*}
Massorna blandas och lösning späds till 100ml.


\section{Buffert 2 pH 7.4}
Enligt utdelade pappret ``Buffertspann'' så är det lämpligaste syra/konjugerade baspar
\ch{H2PO4^-}/\ch{HPO4^{2-}}.\\
Kursboken har en fullständig lösning och anger att molförhållandet 1.5:1 ska gälla
\begin{flalign*}
\frac{[\ch{HPO4^{2-}]}}{[\ch{H2PO4^-}]}&=1.5\\
\end{flalign*}
Man skall lösa 1.5 mol \ch{Na2HPO4} och 1.0 mol \ch{NaH2PO4} i vatten.
Buffertkoncentrationen skall vara 0.1 M och volymen skall vara 100 ml.\\

Molmassorna för respektive förening är
\begin{flalign*}
M(\ch{Na2HPO4})&= 2\cdot22.99+1.01+30.97+4\cdot 16.00\\
               &=141.96 \text{ g}\cdot\text{ mol}^{-1}\\
M(\ch{NaH2PO4})&= 22.99+2\cdot1.01+30.97+4\cdot 16.00\\
               &=119.98 \text{ g}\cdot\text{ mol}^{-1}\\
\end{flalign*}
För att bestämma den erfoderliga substansmängden så har vi formeln
\begin{flalign*}
c&=\frac{n}{V}\iff\\
n(\ch{NaH2PO4})&=c\cdot V=0.1\cdot 0.1 = 0.01\\
n(\ch{Na2HPO4})&=1.5\cdot n(\ch{NaH2PO4})=0.015\\
\end{flalign*}

Massorna som ska vägas upp är
\begin{flalign*}
m(\ch{NaH2PO4})&=n(\ch{NaH2PO4})\cdot M(\ch{NaH2PO4})\\
               &=0.01\cdot 119.98\\
			   &=1.1998 \text{ g}\\
\end{flalign*}
samt
\begin{flalign*}
m(\ch{Na2HPO4})&=n(\ch{Na2HPO4})\cdot M(\ch{Na2HPO4})\\
               &=0.015\cdot 141.96\\
			   &=2.1294\text{ g}\\
\end{flalign*}

\section{Mätverktyg}
Precisionspipett skall användas.


%\vfill\null
%\clearpage
%\columnbreak
%\newpage



%\underbrace{}

% \hspace{1em}

%\begin{enumerate}[label=(\alph*)]
%\end{enumerate}

%$$
%  A = 
%  \begin{bmatrix}
%    1 & 0  & 2i\\
%    2i & 0 &  -4\\
%    -i &  0 & -2i\\
%  \end{bmatrix}
%$$

%\begin{flalign*}
%  A = 
%  \begin{bmatrix}
%    1 & 0  & 2i\\
%    2i & 0 &  -4\\
%    -i &  0 & -2i\\
%  \end{bmatrix}
%\end{flalign*}


%\begin{flalign*}
%\psi(x) = \begin{cases} Ae^{ikx}+Be^{-ikx} &\ \  x<-a \\
%                        Ce^{\kappa x}+De^{-\kappa x} &\ \ -a < x < a\\
%						Fe^{ikx} & \ \ x>a
%       \end{cases}
%\end{flalign*}

%\begin{figure}[H]
%  \includegraphics[width=\linewidth]{odd_finite.eps}
%  \caption{$z_0=0.1\pi,0.5\pi, 3\pi,7\pi$}
%  \label{fig4}
%\end{figure}
\end{document}






\vfill\null
\clearpage
\columnbreak
\newpage










                                     
                                     



