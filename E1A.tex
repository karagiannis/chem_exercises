%\documentclass[twocolumn]{article}
\documentclass[./chem_exercises.tex]{subfiles}
\begin{document}
\begin{multicols}{2}
%\begin{titlepage}
%\maketitle
%\end{titlepage}
\textit{\textbf{Lewis structure and VSEPR theory  } }\\
Lewis structures
\begin{enumerate}
\item Step: Count all the Valence Electrons\\
The total number of valence electrons present in
the molecule of the compound is calculated by adding
the individual valence electrons of each atom.

\item Step: Determine the Central Atom\\
The least electronegative atom is chosen as the central atom of the molecule or ion.
The central metal atom is the one to which all the other atoms will be bonded.
The central atom to be chosen should possess the least subscript in a given molecule.

\item Step: Draw all the Single Bonds to the Central Metal Atom\\
A single bond represents 2 valence electrons, one from each atom contributing to bond formation.
A line represents a single bond.

\item Step: Calculation of lone Pairs of Electrons\\
The number of valence electrons used for bonding in step 3 is subtracted from the total number of valence electrons calculated in step 1.
The remaining electrons are assigned to each atom as lone pair of electrons.

\item Step: Satisfying Octet Configuration for each of the Participating Atoms\\

\begin{enumerate}[label=(\alph*)]
\item For an anion, we need to add extra electrons to the dot structure.
The number of an extra electron that is to be added is always equal to the charge on the anion

\item For a cation, we need to subtract electrons from the total count. 
The number of extra electrons that is to be subtracted is always equal to the charge on the cation.
\end{enumerate}

\end{enumerate}

\hspace{1em}\\

\textit{\textbf{Formal Charge  } }\\
The Formal Charge is countes as
The Formal Charge ($FC$) is 
\begin{flalign*}
FC &=\text{Valence }e^- -\text{Unshared }e^- \\
          &\hspace{1em}-\frac{1}{2}\text{Bonding }e^-\\
\end{flalign*}

\hspace{1em}\\
\hspace{1em}\\

\textit{\textbf{The VSEPR model  } }\\
The VSEPR model of \ch{OF2} is accounted for as follows

\begin{itemize}
\item Count the number of atoms bonded to central atom \\
There are 2 atoms bonded to oxygen which is the central atom
This give the formula \ch{AB2}
\item Count the number of lone pairs on the central atom\\
There are 4 unpaired electrons therfore 2 lone pairs which completes
the formula to \ch{AB2E2}
\end{itemize}
We look into the VSEPR table and find that \ch{AB2E2} is tetrahedral.
\end{multicols}
\vfill\null
\clearpage
%\columnbreak
%\newpage

%\underbrace{}

% \hspace{1em}

%\begin{enumerate}[label=(\alph*)]
%\end{enumerate}

%$$
%  A = 
%  \begin{bmatrix}
%    1 & 0  & 2i\\
%    2i & 0 &  -4\\
%    -i &  0 & -2i\\
%  \end{bmatrix}
%$$

%\begin{flalign*}
%  A = 
%  \begin{bmatrix}
%    1 & 0  & 2i\\
%    2i & 0 &  -4\\
%    -i &  0 & -2i\\
%  \end{bmatrix}
%\end{flalign*}


%\begin{flalign*}
%\psi(x) = \begin{cases} Ae^{ikx}+Be^{-ikx} &\ \  x<-a \\
%                        Ce^{\kappa x}+De^{-\kappa x} &\ \ -a < x < a\\
%						Fe^{ikx} & \ \ x>a
%       \end{cases}
%\end{flalign*}

%\begin{figure}[H]
%  \includegraphics[width=\linewidth]{odd_finite.eps}
%  \caption{$z_0=0.1\pi,0.5\pi, 3\pi,7\pi$}
%  \label{fig4}
%\end{figure}
\end{document}
\vfill\null
\clearpage
\columnbreak
\newpage











                                     
                                     



