%\documentclass[twocolumn]{article}
\documentclass[./chem_exercises.tex]{subfiles}
\begin{document}
\begin{multicols}{2}
%\begin{titlepage}
%\maketitle
%\end{titlepage}

\textit{\textbf{The Lewis structure of $I_3^-$   } }\\

The Lewis structure of $I_3^-$ is\\
\begin{flalign*}
\Big[\ch{
   "\chlewis{90:180:270:}{I}" - "\chlewis{45:135:270:}{I}" - "\chlewis{0:90:270:}{I}" 
}\Big]^-
\end{flalign*}
\begin{enumerate}
\item  Count all the Valence Electrons\\

Iodide belong to group 17 (7A) and have thus 7 valence electrons. The total number
is $3\cdot 7 + 1 = 22$

\item Determine the Central Atom\\

We can only choose Iodine, otherwise it must be the least electronegative atom or the atom with no subscript.

\item Draw all the Single Bonds to the Central Atom\\

Two single bond makes up to 4 electrons

\item Calculation of lone Pairs of Electrons\\

Number of lone pairs are $22-4 = 18$ which gives 6 on each outer Iodine atom which totals 12. Left to
add to the central atom as lone pair electrons are $18-12 = 6$ which is nice divisible by 2.


\item Satisfying Octet Configuration for each of the Participating Atoms\\

The outer Iodine atoms have 6 electrons +2 from the single bond and the center atom has
4 electrons from the two single bonds on each side plus 6 unpaired electrons which totals 10 but one can
also count the bonds as single electrons and in this case it makes 8.
For the outer atoms we count it corresponding single bond as 2 electrons but for the center atom in order
to make it to an octet we are allowed to count each bond as 1 electron.

\begin{enumerate}[label=(\alph*)]
\item For an anion, we need to add extra electrons to the dot structure.\\

This has already been taken to account above.

\end{enumerate}

\end{enumerate}

\hspace{1em}\\
\hspace{1em}\\
\textit{\textbf{Formal Charge  of $I_3^-$ } }\\
\hspace{1em}\\
The Formal Charge  is 
\begin{flalign*}
FC &=\text{Valence }e^- -\text{Unshared }e^- \\
          &\hspace{1em}-\frac{1}{2}\text{Bonding }e^-\\
FC(I_{center})&=7-6-0.5\cdot(2\times 2)=-1\\
FC(I)&=7-6-0.5\cdot(1\times 2)=0\\
\end{flalign*}
The total Formal Charge is
\begin{flalign*}
FC_{tot}&=FC(I_{center})+2\cdot FC(I) = -1
\end{flalign*}
which is also the charge of the ion.\\

\hspace{1em}\\
\hspace{1em}\\
\textit{\textbf{The VSEPR model  } }\\
\hspace{1em}\\
The VSEPR model of $I_3^-$ is accounted for as follows

\begin{itemize}
\item Count the number of atoms bonded to central atom \\
There are 2 atoms bonded to the central Iodine atom.
This give the formula \ch{AB2}
\item Count the number of lone pairs on the central atom\\
There are 6 unpaired electrons therfore 3 lone pairs which completes
the formula to \ch{AB2E3}
\end{itemize}

We look into the VSEPR table and find that \ch{AB2E3} is linear
\end{multicols}
\vfill\null
\clearpage
%\columnbreak
%\newpage



%\underbrace{}

% \hspace{1em}

%\begin{enumerate}[label=(\alph*)]
%\end{enumerate}

%$$
%  A = 
%  \begin{bmatrix}
%    1 & 0  & 2i\\
%    2i & 0 &  -4\\
%    -i &  0 & -2i\\
%  \end{bmatrix}
%$$

%\begin{flalign*}
%  A = 
%  \begin{bmatrix}
%    1 & 0  & 2i\\
%    2i & 0 &  -4\\
%    -i &  0 & -2i\\
%  \end{bmatrix}
%\end{flalign*}


%\begin{flalign*}
%\psi(x) = \begin{cases} Ae^{ikx}+Be^{-ikx} &\ \  x<-a \\
%                        Ce^{\kappa x}+De^{-\kappa x} &\ \ -a < x < a\\
%						Fe^{ikx} & \ \ x>a
%       \end{cases}
%\end{flalign*}

%\begin{figure}[H]
%  \includegraphics[width=\linewidth]{odd_finite.eps}
%  \caption{$z_0=0.1\pi,0.5\pi, 3\pi,7\pi$}
%  \label{fig4}
%\end{figure}
\end{document}

\vfill\null
\clearpage
%\columnbreak
\newpage










                                     
                                     



