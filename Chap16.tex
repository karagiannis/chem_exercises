%\documentclass[twocolumn]{article}
\documentclass[./chem_exercises.tex]{subfiles}
\begin{document}
\begin{multicols}{2}
%\begin{titlepage}
%\maketitle
%\end{titlepage}
\textit{\textbf{Chapter 16 Acid-Base Equilibria and Solubility Equilibria - summary} }\\

\textit{\textbf{ 16.2 Buffer Solutions} }\\
A buffer solution is a solution of a weak acid or a weak base and its salt; both
components must be present. The solution has the ability to resist changes in pH 
upon the addition of small amounts of either acid or base.\\

A buffer solution must contain a relatively large concentration of acid to react with any
\ch{OH^-} ions that are added to it, and it must contain a similar concentration
of base to react with any added \ch{H^+} ions. Furthermore, the acid and the base components of 
the buffer must not consume each other in a neutralization reaction.
These requirements are satisfied by an acid-base conjugate pair, for example
a weak acid and its conjugate base(supplied by a salt) or a weak base
and its conjugate acid (supplied by a salt).


\end{multicols}


%\vfill\null
%\clearpage
%\columnbreak
%\newpage



%\underbrace{}

% \hspace{1em}

%\begin{enumerate}[label=(\alph*)]
%\end{enumerate}

%$$
%  A = 
%  \begin{bmatrix}
%    1 & 0  & 2i\\
%    2i & 0 &  -4\\
%    -i &  0 & -2i\\
%  \end{bmatrix}
%$$

%\begin{flalign*}
%  A = 
%  \begin{bmatrix}
%    1 & 0  & 2i\\
%    2i & 0 &  -4\\
%    -i &  0 & -2i\\
%  \end{bmatrix}
%\end{flalign*}


%\begin{flalign*}
%\psi(x) = \begin{cases} Ae^{ikx}+Be^{-ikx} &\ \  x<-a \\
%                        Ce^{\kappa x}+De^{-\kappa x} &\ \ -a < x < a\\
%						Fe^{ikx} & \ \ x>a
%       \end{cases}
%\end{flalign*}

%\begin{figure}[H]
%  \includegraphics[width=\linewidth]{odd_finite.eps}
%  \caption{$z_0=0.1\pi,0.5\pi, 3\pi,7\pi$}
%  \label{fig4}
%\end{figure}
\end{document}

\vfill\null
\clearpage
\columnbreak
\newpage










                                     
                                     



