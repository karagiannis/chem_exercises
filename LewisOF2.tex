%\documentclass[twocolumn]{article}
\documentclass[./chem_exercises.tex]{subfiles}
\begin{document}
\begin{multicols}{2}
%\begin{titlepage}
%\maketitle
%\end{titlepage}


\textit{\textbf{The Lewis structure of \ch{OF2}  } }\\

The Lewis structure of \ch{OF2} is\\
\hspace{1em}\\
\ch{
  !(\elconf{O})( "\chlewis{0.90:180.270:}{O}" ) +
  !(\elconf{F})( "2"  "\chlewis{0.90:180:270:}{F}" ) 
   ->
   "\chlewis{90:180:270:}{F}" - "\chlewis{90:270:}{O}" - "\chlewis{0:90:270:}{F}"
}\par
\hspace{1em}\\

\begin{enumerate}

\item  Count all the Valence Electrons\\

Total number of valence electrons are 
\begin{flalign*}
O&\rightarrow 6e^-\\
F&\rightarrow 7e^-\\
Total &= 2\cdot 7+6 = 20\\
\end{flalign*}

\item Determine the Central Atom\\

The oxygen is the central atom because it has no subscripts and is the least electro negative.

\item Draw all the Single Bonds to the Central Atom\\

This makes two single bonds of each side of the oxygen atom

\item Calculation of lone Pairs of Electrons\\

Number of bonding electrons and unshared electrons are
2 single bonds $2\times 2 =4$ plus number of unshared electrons
are $6+4+6 = 16$. Total electrons bonding + unshared is 20.

\item Satisfying Octet Configuration for each of the Participating Atoms\\

6 unpaired electrons on each of the fluoride atoms plus a single bond each completes the
octet for the fluoride atoms which makes 12.
Number of electrons left is 8 and the oxygen has 4 electrons from the two single bonds where each single
bond counts for 2 electrons then we have to add the remaining 4 on the oxygen as unpaired electrons.
\end{enumerate}
\hspace{1em}\\
\hspace{1em}\\
\textit{\textbf{Formal Charge  of \ch{OF2} } }\\
\hspace{1em}\\
 The Formal Charge of \ch{OF2} is 
\begin{flalign*}
FC &=\text{Valence }e^- -\text{Unshared }e^- \\
          &\hspace{1em}-\frac{1}{2}\text{Bonding }e^-\\
FC(O)&=6-4-0.5\cdot(2\times 2)=0\\
FC(F)&=7-6-0.5\cdot(1\times 2)=0\\
\end{flalign*}

\hspace{1em}\\
\hspace{1em}\\
\textit{\textbf{The VSEPR model of \ch{OF2}  } }\\
\hspace{1em}\\
The VSEPR model of \ch{OF2} is accounted for as follows

\begin{itemize}
\item Count the number of atoms bonded to central atom \\
There are 2 atoms bonded to oxygen which is the central atom
This give the formula \ch{AB2}
\item Count the number of lone pairs on the central atom\\
There are 4 unpaired electrons therfore 2 lone pairs which completes
the formula to \ch{AB2E2}
\end{itemize}
We look into the VSEPR table and find that \ch{AB2E2} is tetrahedral.
\end{multicols}
\vfill\null
\clearpage
%\columnbreak
%\newpage

%\underbrace{}

% \hspace{1em}

%\begin{enumerate}[label=(\alph*)]
%\end{enumerate}

%$$
%  A = 
%  \begin{bmatrix}
%    1 & 0  & 2i\\
%    2i & 0 &  -4\\
%    -i &  0 & -2i\\
%  \end{bmatrix}
%$$

%\begin{flalign*}
%  A = 
%  \begin{bmatrix}
%    1 & 0  & 2i\\
%    2i & 0 &  -4\\
%    -i &  0 & -2i\\
%  \end{bmatrix}
%\end{flalign*}


%\begin{flalign*}
%\psi(x) = \begin{cases} Ae^{ikx}+Be^{-ikx} &\ \  x<-a \\
%                        Ce^{\kappa x}+De^{-\kappa x} &\ \ -a < x < a\\
%						Fe^{ikx} & \ \ x>a
%       \end{cases}
%\end{flalign*}

%\begin{figure}[H]
%  \includegraphics[width=\linewidth]{odd_finite.eps}
%  \caption{$z_0=0.1\pi,0.5\pi, 3\pi,7\pi$}
%  \label{fig4}
%\end{figure}
\end{document}

\vfill\null
\clearpage
%\columnbreak
%\newpage










                                     
                                     



