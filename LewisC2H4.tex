%\documentclass[twocolumn]{article}
\documentclass[./chem_exercises.tex]{subfiles}
\begin{document}
\begin{multicols}{2}
%\begin{titlepage}
%\maketitle
%\end{titlepage}
\textit{\textbf{The Lewis structure of \ch{C2H4}   } }\\

The Lewis structure of \ch{C2H4} is
\begin{flalign*}
\chemfig{H-[1]C(-[3]H)(=C(-[1]H)(-[7]H))}
\end{flalign*}
which is found as follows:
\begin{enumerate}
\item  Count all the Valence Electrons\\

There are 2 C-atoms each containing 4 valence electrons and
4 H-atoms containg one electron each, so total valence electrons
are $2\cdot 4 +4\cdot 1 = 12$

\item Determine the Central Atom\\

Hydrogen has lesser electronegativity (=ability to attract electrons)
but it usually occupy terminal positions and hence the C-atoms
are the central atoms.


\item Draw all the Single Bonds to the Central Atom\\

5 single bonds is 10 electrons.

\item Calculation of lone Pairs of Electrons\\

There are two electrons reminaing $12-10 = 2$


\item Satisfying Octet Configuration for each of the Participating Atoms\\

The two electrons which were remaing are used as an extra bond

\begin{enumerate}[label=(\alph*)]
\item For an anion, we need to add extra electrons to the dot structure.
The number of an extra electron that is to be added is always equal to the charge on the anion

\item For a cation, we need to subtract electrons from the total count. 
The number of extra electrons that is to be subtracted is always equal to the charge on the cation.
\end{enumerate}

Neither applies here.

\end{enumerate}

\hspace{1em}\\
\hspace{1em}\\
\textit{\textbf{Formal Charge  of \ch{C2H4}  } }\\
\hspace{1em}\\
We need the number of  of bonding electrons. The C-atoms have
one double bound which is 4 electrons + 2 single bonds each for the two
hydrogens.
\begin{flalign*}
FC &=\text{Valence }e^- -\text{Unshared }e^- \\
          &\hspace{1em}-\frac{1}{2}\text{Bonding }e^-\\
FC(C)&=4-0-0.5\cdot(1\times 4 +2\times 2)=0\\
FC(H)&=1-0-0.5\cdot(1\times 2)=0\\
\end{flalign*}
The sum of the formal charges for all 6 atoms is 0.


\hspace{1em}\\
\hspace{1em}\\
\textit{\textbf{The VSEPR model  } }\\
\hspace{1em}\\

The VSEPR model of \ch{C2H4} is acccounted as follows
\begin{itemize}
\item Count the number of atoms bonded to central atom \\
There are 2 C central atoms bonded to the central C-atoms
but according to the theory one should pick one central atom.
This gives the formula \ch{AB3}
\item Count the number of lone pairs on the central atom\\
There are no unpaired electrons hence \ch{AB3} is the final answer.
\end{itemize}
We look into the VSEPR table and find that \ch{AB3} is the trogonal planar molecular
geometry.
\end{multicols}
\vfill\null
\clearpage
%\columnbreak
%\newpage



%\underbrace{}

% \hspace{1em}

%\begin{enumerate}[label=(\alph*)]
%\end{enumerate}

%$$
%  A = 
%  \begin{bmatrix}
%    1 & 0  & 2i\\
%    2i & 0 &  -4\\
%    -i &  0 & -2i\\
%  \end{bmatrix}
%$$

%\begin{flalign*}
%  A = 
%  \begin{bmatrix}
%    1 & 0  & 2i\\
%    2i & 0 &  -4\\
%    -i &  0 & -2i\\
%  \end{bmatrix}
%\end{flalign*}


%\begin{flalign*}
%\psi(x) = \begin{cases} Ae^{ikx}+Be^{-ikx} &\ \  x<-a \\
%                        Ce^{\kappa x}+De^{-\kappa x} &\ \ -a < x < a\\
%						Fe^{ikx} & \ \ x>a
%       \end{cases}
%\end{flalign*}

%\begin{figure}[H]
%  \includegraphics[width=\linewidth]{odd_finite.eps}
%  \caption{$z_0=0.1\pi,0.5\pi, 3\pi,7\pi$}
%  \label{fig4}
%\end{figure}
\end{document}

\vfill\null
\clearpage
%\columnbreak
%\newpage










                                     
                                     



